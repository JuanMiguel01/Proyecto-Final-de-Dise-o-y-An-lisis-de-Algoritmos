\documentclass{article}
\usepackage{amsmath}
\usepackage{amssymb}
\title{Reducción de un problema de los 21 de karp a un problema propuesto en este caso Batalla Naval}
\title{Reducción de Partición a Bin Packing y Bin Packing a Batalla Naval}
\author{Juan Miguel Pérez Martínez c412}

\begin{document}

\maketitle

\section{Reducción de Partición a Bin Packing}

El problema de Bin Packing es NP-completo. Más específicamente:

\textbf{Teorema 8.1.} Es NP-completo decidir si una instancia de Bin Packing admite una solución con dos contenedores.

\textbf{Demostración:} Reducimos desde Partición, la cual sabemos que es NP-completa. Recordemos que en el problema de Partición, se nos dan $n$ números $c_1, \dots, c_n \in \mathbb{N}$ y se nos pide decidir si existe un conjunto $S \subseteq \{1, \dots, n\}$ tal que
\[
\sum_{i \in S} c_i = \sum_{i \notin S} c_i.
\]
Dada una instancia de Partición, creamos una instancia para Bin Packing estableciendo $s_i = \frac{2c_i}{\sum_{j=1}^{n} c_j} \in (0, 1]$ para $i = 1, \dots, n$. Obviamente, dos contenedores son suficientes si y solo si existe un $S \subseteq \{1, \dots, n\}$ tal que
\[
\sum_{i \in S} c_i = \sum_{i \notin S} c_i.
\]

Esto nos permite derivar una cota inferior sobre la aproximabilidad de Bin Packing.

\textbf{Corolario 8.2.} No existe un algoritmo de aproximación con factor $\rho < \frac{3}{2}$ para Bin Packing a menos que $P = NP$.

\section{Reducción de Bin Packing a Batalla Naval}

El problema de Batalla Naval es NP-completo.

\textbf{Definición:} El problema se define de la siguiente manera. Un rompecabezas de Batalla Naval consiste en una cuadrícula de rompecabezas, un conteo de filas y columnas, y una flota que contiene un cierto número de barcos de longitudes variables. La cuadrícula inicial está parcialmente llena con segmentos de barcos o agua. El objetivo del rompecabezas es demostrar que los barcos proporcionados pueden colocarse en la cuadrícula de acuerdo con las siguientes cuatro condiciones:
\begin{itemize}
    \item Todos los barcos de la flota se colocan en la cuadrícula;
    \item Se respetan las indicaciones en la cuadrícula inicial;
    \item Ningún barco ocupa casillas adyacentes (ni ortogonal ni diagonalmente);
    \item El número de segmentos de barco en la columna (fila) $i$ es igual al valor $i$-ésimo del conteo de columnas (filas).
\end{itemize}

Podemos reducir el problema de Bin Packing al rompecabezas de Batalla Naval de la siguiente manera:

Entrada: $n$ enteros positivos $a_1, \dots, a_n$ (los objetos), dos enteros $C$ (la capacidad) y $B$ (el número de contenedores) tal que $a_1 + \dots + a_n = CB$. 

\textbf{Pregunta:} ¿Se puede particionar $a_1, \dots, a_n$ en $B$ subconjuntos, cada uno de los cuales tiene una suma total exactamente igual a $C$?

Transformamos una instancia de Bin Packing en un rompecabezas de Batalla Naval reservando una franja vertical de longitud $a_i$ por cada par de objeto $a_i$ y contenedor $b \in \{1, \dots, B\}$. Todas las demás casillas se llenan inicialmente con agua. Nótese que esta reducción asume que los números $a_1, \dots, a_n$ están representados en unario, ya que las franjas mismas son representaciones unarias de estos números. Esta suposición no afecta el resultado, ya que el problema de Bin Packing sigue siendo NP-completo si sus números se representan en unario. Esta propiedad hace que el problema sea NP-completo fuerte.

La idea es que cada franja abierta representa la posibilidad de que el objeto $a_i$ sea colocado en el contenedor $b$. El objeto mismo está representado por un barco de longitud $a_i$. Para asegurarnos de que cada objeto $a_i$ se coloque en solo un contenedor, tenemos 1's en el conteo de filas correspondiente. Esto también asegura que cada barco de longitud $a_i$ se coloque en una franja de longitud exactamente $a_i$.

Esta reducción prueba que Batalla Naval es NP-difícil. Es fácil ver que una conjetura no determinista puede verificarse como una solución en tiempo polinomial. Por lo tanto, Batalla Naval es NP-completo.




\section*{Concluciones del primer intento de reduccion:}
Esta reduccion no me convence ya que en la primera parte  de Partición a Bin Packing solo reduzco para una instancia del problema por lo que seguí investigando y llegué a la realiacion de otra reducción 

\section{Reducción de 3-SAT a 3,4-SAT}
Comience con cualquier instancia de 3-SAT. Para cada variable \(x\) que aparece en más de tres cláusulas (por ‘aparece’ nos referimos a que ella o su complemento está en la cláusula), realice el siguiente procedimiento: Suponga que \(x\) aparece en \(k\) cláusulas. Cree \(k\) nuevas variables \(x_1, \ldots, x_k\) y reemplace la \(i\)-ésima ocurrencia de \(x\) con \(x_i\), \(i = 1, \ldots, k\). Añada la cláusula \(\{x_i \lor \neg x_{i+1}\}\) para \(i=1, \ldots, k-1\) y la cláusula \(\{x_k \lor \neg x_1\}\). En la nueva instancia, la cláusula \(\{x_i \lor \neg x_{i+1}\}\) implica que si \(x_i\) es falso, \(x_{i+1}\) también debe ser falso. La estructura cíclica de las cláusulas, por lo tanto, obliga a que los \(x_i\) sean todos verdaderos o todos falsos, por lo que la nueva instancia es satisfacible si la original lo es. Además, la transformación requiere tiempo polinomial.

\textbf{Teorema 2.1}. La satisfacibilidad booleana es NP-completa cuando se restringe a instancias con 2 o 3 variables por cláusula y como máximo 3 ocurrencias por variable.

Un corolario interesante es:

\textbf{Corolario 2.2}. Para cualquier \(s \geq 3\), o bien (i) toda expresión booleana con exactamente 3 variables por cláusula y no más de \(s\) ocurrencias por variable es satisfacible, o (ii) 3,\(s\)-SAT es NP-completo. 

\textbf{Prueba}. Suponga que (i) no se cumple, por lo que existe una expresión insatisfacible en las variables \(x_1, y, z, \ldots\). Sin pérdida de generalidad, la primera cláusula de la expresión incluye una ‘\(x\)’ sin complementar. Sea \(B\) el resto de la expresión; claramente podemos asumir que \(B\) es satisfacible. \(B\) ahora tiene las propiedades de que \(x\) aparece como máximo \(s-1\) veces, y que solo puede ser satisfecho cuando \(x\) es falso.

Ahora considere una instancia arbitraria de 3-SAT y realice el procedimiento en la construcción del Teorema 2.1. Para la \(i\)-ésima cláusula, \((a \lor b)\) que contiene dos variables, añada una \(i\)-ésima copia de \(B\) usando las variables \(x_i, y, z, \ldots\), y cambie la cláusula a \((a \lor b \lor x_i)\). Así que si (i) es falso, entonces (ii) es verdadero. Note que (i) y (ii) no pueden ser ambos verdaderos a menos que \(P=NP\).

\textbf{Teorema 2.3}. 3,4-SAT es NP-completo. 

\textbf{Prueba}. Lo único que falta en la construcción del Teorema 2.1 es que las cláusulas \((x_i \lor x_{i+1})\) contienen solo dos variables. Para cada una de estas cláusulas, introduzca una nueva variable \(y_i\), de modo que la cláusula se convierta en \((x_i \lor x_{i+1} \lor y_i)\). Ahora note que podemos forzar a que cada \(y_i\) sea verdadero mediante las siguientes cláusulas en las que \(y_i\) aparece solo tres veces. La construcción es, sospechamos, lo más pequeña posible. Las primeras tres cláusulas requieren que \(y_i\) sea verdadero si cualquiera de los pares \(a_i, b_i\) son ambos falsos; las otras diez cláusulas fuerzan a que esto suceda:

\[
\{y_i \lor a_i \lor b_i\}, \quad i= 1, \ldots, 3,
\]
\[
\{d_i \lor D_i \lor b_i\}, \quad \{d_i \lor O_i \lor b_i\}, \quad \{d_i \lor a_i \lor b_i\}, \quad j=1, \ldots, 3,
\]
\[
\{G_i \lor D_i \lor b_i\}.
\]

En la reducción real, habría una copia de lo anterior para cada \(y_i\); hemos suprimido el subíndice ‘i’ de cada variable para mayor legibilidad. Si la instancia original de 3-SAT tiene \(m\) cláusulas, la instancia 3,4-SAT tendrá \(m + 3m + 39m = 43m\) cláusulas. El Corolario 2.2 y el algoritmo en la Sección 3 demuestran que no hay un nivel intermedio de complejidad en los problemas 3,\(s\)-SAT entre lo virtualmente trivial y lo NP-completo. El único caso no resuelto es 3,3-SAT.

\section*{Teorema 2.4.1}
Toda instancia de \( (r,r)\)-SAT es satisfactoria.

\textbf{Demostración}. Denotemos las variables por \( x_1, x_2, \ldots, x_n \) y las cláusulas por \( c_1, \ldots, c_m \), donde \( m \leq n \) (y \( m = n \) solo si cada variable aparece en exactamente \( r \) cláusulas). Sea \( A = A_1, \ldots, A_m \) la familia de conjuntos (no necesariamente distintos) de los \( x_i \) definidos por \( A_i = \{x_j \mid x_j \in c_i \text{ o } x_j \in c_j\} \).

Consideremos cualquier unión de \( k \) de los conjuntos \( A_i \). Dado que cada \( A_i \) contiene \( r \) elementos distintos y ningún \( x_j \) está contenido en más de \( r \) conjuntos, la unión contiene al menos \( k \) elementos distintos. Entonces, por el Teorema de Philip Hall, existe un sistema de representantes distintos de \( A \). Es decir, existe una inyección de las cláusulas \( c_i \) a las variables \( x_j \) tal que cada cláusula contiene la variable (o el complemento de la variable) a la que está asignada. Es trivial, dado tal mapeo, satisfacer cada cláusula, y por lo tanto, la instancia de \( (r,r) \)-SAT.

\section*{Conjetura 2.5}
Si \( s \leq 2^{r-1} - 1 \), entonces toda instancia de \( (r,s) \)-SAT es satisfactoria.

El significado intuitivo del término \( 2^{r-1} - 1 \) es el número mínimo de cláusulas de tamaño \( r \) requerido para forzar que una variable sea verdadera, si las variables ‘forzadoras’ están restringidas a la mitad de sus posibles valores de verdad. Por ejemplo, solo 7 de las 16 combinaciones de verdad de \( y_1, \ldots, y_4 \) restringen a un valor en las cláusulas \( \{t \lor y_1 \lor y_2, t \lor y_3 \lor y_4\} \). En el caso \( (3,4) \), la fracción análoga es 37/64, que es más de 1/2, por lo que la ‘ola’ de implicaciones de los \( y \) a \( z \) se vuelve más fuerte y es posible forzar una contradicción.
\section{Existencia de una reducción parsimoniosa de 3-SAT a nuestro problema de decisión de Battleships}

\textbf{Teorema:} Existe una reducción  de 3-SAT a BATTLESHIPS.

\textbf{Demostración:} Probaremos esto proporcionando dos reducciones: una de 3-SAT a un problema intermedio 3, \{3, 4\}-SAT, y la otra de 3, \{3, 4\}-SAT a BATTLESHIPS. 3-SAT es el conjunto de fórmulas booleanas satisfacibles que están en forma normal conjuntiva y tienen tres literales por cláusula. Como es habitual, asumimos que ninguna variable aparece dos veces en una sola cláusula. 3, \{3, 4\}-SAT es el subconjunto de 3-SAT que contiene todas las fórmulas satisfacibles con exactamente tres variables por cláusula y cada variable aparece tres o cuatro veces. Decidir si una instancia de 3, \{3, 4\}-SAT es satisfacible fue probado como NP-completo anteriormente; el resultado se estableció mediante una reducción  de 3-SAT.

Dedicaremos el resto de esta demostración a una reducción  de 3, \{3, 4\}-SAT a BATTLESHIPS. Para ello, sea $\phi \equiv C_1 \land \ldots \land C_m$ una instancia de 3, \{3, 4\}-SAT sobre las variables $x_1, \ldots, x_n$. Asociamos con cada cláusula $C_i$ y variable $x_j$ una región $X_{ij}$. Además, cada cláusula $C_i$ (variable $x_j$, respectivamente) está asociada con una región marcada $Y_i$ ($Z_j$, respectivamente). Construimos el rompecabezas colocando dispositivos en las regiones marcadas con $X$, $Y$ y $Z$ junto con la flota y los conteos de filas y columnas, usando $\phi$. Terminamos la construcción fijando los valores de conteo restantes. Procedemos de la siguiente manera:

\begin{itemize}
    \item Para $X_{ij}$: si $x_j$ aparece positivamente en $C_i$, colocamos un dispositivo específico; si $x_j$ aparece negativamente en $C_i$, colocamos otro dispositivo; de lo contrario, si $x_j$ no aparece en $C_i$, colocamos un dispositivo de agua. Todos los dispositivos tienen una altura de $8i + 3$. Si $x_j$ aparece en $C_i$, añadimos un barco de longitud 1, el barco $X_{ij}$, a la flota.
    \item Para $Y_i$: colocamos un dispositivo de altura $8i + 3$; añadimos un barco de longitud $4i + 1$, un barco de longitud $4i$ y un barco de longitud 1. Los barcos añadidos los llamamos barcos $Y_i$. Este dispositivo también determina el conteo de filas para las filas que intersectan $Y_i$.
    \item Para $Z_j$: si $x_j$ aparece cuatro veces en $\phi$, colocamos un dispositivo específico y añadimos un barco de longitud 4; de lo contrario, si $x_j$ aparece tres veces, colocamos otro dispositivo y añadimos un barco de longitud 3. El barco añadido lo llamamos barco $Z_j$. Este dispositivo también determina el conteo de filas para las filas que intersectan $Z_j$.
    \item Concluimos la construcción asignando valores de conteo a las filas más bajas y las columnas más a la derecha, que hasta ahora permanecían abiertas:
\end{itemize}

\begin{table}[h!]
    \centering
    \begin{tabular}{|c|c|}
        \hline
        \textbf{Conteo de filas} & \textbf{Conteo de columnas} \\
        \hline
        $9m + 1$ & 0 \\
        $3n + 1$ & $\sum_{1 \leq i \leq m} (4i + 1)$ \\
        $9m + 2$ & $n_4$ \\
        $3n + 2$ & 0 \\
        $9m + 3$ & $n$ \\
        $3n + 3$ & $\sum_{1 \leq i \leq m} (4i + 1)$ \\
        $9m + 4$ & $n$ \\
        $9m + 5$ & $n$ \\
        \hline
    \end{tabular}
    \caption{Conteo de filas y columnas.}
\end{table}

donde $n_4$ es igual al número de variables que aparecen cuatro veces en $\phi$.

Basta con mostrar que el número de asignaciones de verdad que satisfacen $\phi$ es igual al número de soluciones del rompecabezas de Battleships anterior. De hecho, mostraremos que cada asignación de verdad satisfactoria de $\phi$ corresponde exactamente a una solución de la reducción de $\phi$, y viceversa.

Supongamos que $\phi$ tiene una asignación de verdad satisfactoria $t: \{x_1, \ldots, x_m\} \rightarrow \{true, false\}$. Entonces resolveremos el rompecabezas reducido usando $t$ de la siguiente manera. Para cada variable $x_j$ que aparece en la cláusula $C_i$, colocamos el barco $X_{ij}$ en la ranura noreste, si $x_j$ aparece positivamente en $C_i$ y $t(x_j) = true$; en la ranura suroeste, si $x_j$ aparece positivamente en $C_i$ y $t(x_j) = false$; en la ranura sureste, si $x_j$ aparece negativamente en $C_i$ y $t(x_j) = true$; en la ranura noroeste, si $x_j$ aparece negativamente en $C_i$ y $t(x_j) = false$. Es inmediato que $t$ está codificado de manera única por la posición de los barcos $X_{ij}$.

Si la asignación de verdad de $x_j$ contribuye a la verdad de su cláusula $C_i$ depende de si $x_j$ aparece negativamente o positivamente en $C_i$. Por ejemplo, si $x_j$ aparece negativamente en $C_i$ y se asigna como verdadero, entonces el barco $X_{ij}$ debe colocarse en la ranura abierta derecha. En consecuencia, se coloca en la fila inferior. Un barco $X_{ij}$ se coloca en la ranura abierta inferior si no contribuye a que $C_i$ sea verdadero. Dado que $t$ es una asignación satisfactoria, al menos uno de los literales de $C_i$ es verdadero. En el rompecabezas de Battleships, esto corresponde a al menos un barco $X_{ij}$ colocado en la fila superior.
El dispositivo en $Y_i$ y su conteo pueden satisfacerse para cualquier disposición de los barcos $X_{ij}$ donde el número de barcos en la fila superior (inferior) sea al menos uno (dos). En cuanto al número de barcos $X$ en la fila superior, hacemos una distinción de casos. Supongamos que la fila superior contiene tres barcos $X$, entonces el conteo de filas correspondiente ya está satisfecho. El resto del dispositivo $Y_i$ solo puede resolverse organizando los barcos $Y_i$ de manera específica.

Si la fila superior contiene dos barcos $X$, entonces hay varias posibles soluciones. Descartamos una de las soluciones exigiendo que la primera y tercera columna de los dispositivos $Y$ contengan $\sum_{1 \leq i \leq m} 4i + 1$ segmentos de barco. Si se eligiera un patrón incorrecto, nunca se podría resolver el rompecabezas debido a esta restricción.

Supongamos que la fila superior contiene solo un barco $X$, entonces debe respetarse la solución única del rompecabezas, que garantiza que los dispositivos $Y_i$ se resuelvan correctamente.

A partir de la definición de $t$, se sigue que cada variable $x_j$ se asigna a un valor de verdad único. Si $t(x_j) = \text{true}$ (es decir, verdadero), entonces el barco se coloca en la columna derecha. Si $t(x_j) = \text{false}$ (falso), entonces el barco se coloca en la columna izquierda.

Ahora, supongamos que $t(x_j) = \text{true}$, entonces el conteo de columnas derecho de 3 (o 4, en caso de que $x_j$ aparezca cuatro veces en $\phi$) se satisface; el conteo de columnas izquierdo de 3 (o 4) se satisface colocando el barco $Z_j$ en la ranura izquierda, y así sucesivamente. De esta manera, el posicionamiento de los barcos asegura que la asignación de verdad se refleje correctamente en el tablero del rompecabezas.

Esto completa la construcción del rompecabezas de BATTLESHIPS y muestra que cualquier solución válida al rompecabezas corresponde a una asignación de verdad satisfactoria para $\phi$. Por lo tanto, hemos demostrado la existencia de una reducción  de 3-SAT a BATTLESHIPS.
\end{document}
